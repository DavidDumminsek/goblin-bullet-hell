\documentclass{mall}

\newcommand{\version}{Version 1.0}
\author{David Dumminsek, \url davdu153@student.liu.se\\
Haris Basic, \url harba466@student.liu.se }
\title{Gruppkontrakt}
\date{2022-11-14}
\rhead{}


\begin{document}
\projectpage
\section{Förutsättningar}
\label{prereq}
\section{Hur vi arbetar tillsammans}


\begin{itemize}
\item \textbf{Vilka tider arbetar vi, och vilka tider är vi nåbara utöver detta?}

Vi arbetar för mestadels självständigt hemifrån runt tiden 9-17, att vi enbart är nåbara vid denna tiden. 

\item \textbf{Hur kommunicerar vi med varandra? Vilka verktyg/kanaler använder vi? Hur och när är det okej att vi avbryter varandra?}

Vi kommer använda discord och sms för att kommunicera med varandra. Vi bör bara förvänta oss svar vid ansatt arbetstider.

\item \textbf{Hur gör vi för att ge varandra möjlighet att framföra åsikter och tankar om uppgifter och idéer till arbetet?}

Att det är okej att tala fritt.

\item \textbf{Hur ofta tar vi paus? Ska vi hjälpas åt att påminna varandra om att ta paus?}

Man ansvarar själv för sina egna pauser. Man kan ta använding av en timer för att visa att det är tid för en paus.
Vi vill sikta på en kvarts paus varannan timme.

\item \textbf{Arbetar vi tillsammans med uppgifter, eller var för sig?}

Vi delar olika delar av projektet och arbeter på dem var för sig.

\item \textbf{Hur bestämmer vi vem som gör vad?}

Vi spenderar en dag till bara att dela upp projektet enligt delar av designspecifikationen.
Vi lägger också till eventuella uppgifter om delar upp dem så fort vi kan.

\item \textbf{Hur specifierar vi vad som ingår i varje uppgift, och när den är klar?}

Vi från en början delar upp projektet i alla klasser som kommer behövas byggas. Utifrån dessa klasser kan större delar bli uppgifter.
Uppgifterna kommer sammlas i ett mer detaljerat dokument och tilldelas via någon valfri programvara som t.ex jira eller trello. 

\item \textbf{Hur snabbt förväntar vi oss att en uppgift kan vara klar?}

Inom runt en vecka, men kan avvika beroende på uppgift.

\item \textbf{Hur håller vi reda på att uppgifter vi identifierat inte glöms bort?}

Vi kommer använda dokumentet och någon av webbsidorna tidigare nämnt.

\end{itemize}

\end{document}
