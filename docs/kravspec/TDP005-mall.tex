\documentclass{TDP005mall}



\newcommand{\version}{Version 1.1}
\author{David Dumminsek, \url{davdu153@student.liu.se}\\
  Haris Basic, \url{harba466@student.liu.se}\\
  }
\title{Kravspecifikation}
\date{2022-11-15}
\rhead{David Dumminsek\\
Haris Basic\\
}



\begin{document}
\projectpage
\section{Revisionshistorik}
\begin{table}[!h]
\begin{tabularx}{\linewidth}{|l|X|l|}
\hline
Ver. & Revisionsbeskrivning & Datum \\\hline
1.0 & Skapad för studenter att använda som mall till
kommande dokumentinlämningar & 140908 \\\hline
\end{tabularx}
\end{table}

\section{Dogeater: The Goblin Ace}
Ett enkelt ''bullet hell'' spel.
\subsection{Spelidé}
Spelet kommer vara ett klassiskt ''bullet hell'' spel där spelaren ska undvika en stor mängd projektiler och fiender tills slutet av banan har nåtts. 
Spelaren kommer styra ett stridsflygplan i en vertikalt skrollande värld.

Fiender kommer skapas och långsamt visa sig själv när banan 
har skrollat upp till dem. Det kommer finnas flera olika fiendetyper där de har olika rörelse och attack mönster. 
Fienderna kommer skjuta farliga projektiler där om dem kolliderar med spelare så förlorar spelaren ett liv och börjar om banan från början.
Om spelaren skulle dö utan några liv kvar skickat spelaren tillbaka till menyn. 
Spelaren ska också ha möjlighet att skjuta projektiler där varje projektil skadar fiender och om en fiende tar tillräcklig skada så dör den
och spelaren får poäng som bygger upp till ett nytt liv.

\subsection{Målgrupp}
Spelet är riktade till spelare som gillar utmaningar och inte är rädd för ett oförlåtande spel.
%Spelet innehåller mild våld där det värsta är att en fiende sprängs. 
%Så enligt ESRB så kommer detta spelet vara fullt lämpligt för barn över 10 år.
\subsection{Spelupplevelse}
Det är uppfyllelsen man upplever efter man har klarat en svår bana. 
Spänningen man känner när man knappt undviker en ogudlig mängd projektiler och vet om man 
blir träffat så måste man börja om banan.
\subsection{Spelmekanik}
Menyn navigeras med piltangenterna och för att trycka på en meny knapp används spacebar.
\begin{table}[h]
\begin{tabularx}{\linewidth}{|l|X|}
\hline
  Tangent & handling \\\hline
  $\uparrow$ & Rör spelaren uppåt \\\hline
  $\leftarrow$ & Rör spelaren åt vänster \\\hline
  $\leftarrow$ + $\uparrow$ & Rör spelaren diagonalt upp åt vänster \\\hline
  $\leftarrow$ + $\downarrow$ & Rör spelaren diagonalt ner åt vänster \\\hline
  $\rightarrow$ & Rör spelaren åt höger \\\hline
  $\rightarrow$ + $\uparrow$ & Rör spelaren diagonalt up till höger \\\hline
  $\rightarrow$ + $\downarrow$ & Rör spelaren diagonalt ner till höger \\\hline
  $\downarrow$ & Rör spelaren neråt \\\hline
  spacebar & Skjuter en projektil \\\hline  
\end{tabularx}
\end{table}
\section{Regler}


\end{document}
