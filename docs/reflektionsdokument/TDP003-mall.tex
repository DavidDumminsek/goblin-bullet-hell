\documentclass{TDP003mall}



\newcommand{\version}{Version 0.1}
\author{David Dumminsek, \url{davdu153@student.liu.se}\\}
\title{Reflektionsdokument}
\date{2022-12-15}
\rhead{David Dumminsek}



\begin{document}
\projectpage
\section{Revisionshistorik}
\begin{table}[!h]
\begin{tabularx}{\linewidth}{|l|X|l|}
\hline
Ver. & Revisionsbeskrivning & Datum \\\hline
0.1 & Första utkast & 22-12-15\\\hline
\end{tabularx}
\end{table}


\section{Inledning}
Projektet var ett spel gjort i programmerings språket c++ med hjälp av grafik biblioteket
SFML. Det var meningen att göra spelet i projektgrupper av två personer. Tyvärr så behövdes detta
projekt göras ensamt då den ena projektmedlemmen ''försvann'' utan några spår. 

Så alla bidrag till projektet var det jag själv som gjorde då vi inte hade producerat någon kod 
tillsammans. Jag var också helt ny till c++ så det har verkligen kunnat ha gått bättre.

Slutprodukten blev ett väldigt enkelt ''bullethell'' spel utan några animationer eller komplexa 
mönster av projektiler. 
Det blev fortfarande ett fullt funktionellt spel som jag kan vara stolt över.

\section{Tekniker}
Under projektets gång fick jag stöta på en hel del olika tekniker.
\subsection{Cmake}
Det var stort fokus på att projektet skulle vara objekt orienterat, alltså uppbyggt av
många olika klasser. I c++ leder det till att det blir en del så kallade header filer. Som 
är filer med deklarationer av klasser, funktioner och variabler. God praxis är då att göra en
klass per header fil och när man ska kompilera så måste alla filer länkas. 
Så om man har en del klasser och bibliotek så kommer kommandot för kompileringen bli rätt så stor.

Det cmake gör är helt enkelt en makefile som kompilerar allting. 
Så efter man har gjort en ändring är det bara att köra kommandot \textbf{make} så körs make filen och allt kompileras.

Jag hade det väldigt svårt att få cmake att fungera först då jag inte förstod hur man 
skulle koppla de biblioteken jag skulle använda. Först trodde jag det skulle räcka om jag bara
berättade vad länkning flaggorna var men det funkade inte så bra. Så jag googlade och försökte
en del olika metoder jag såg på stackoverflow, helt enkelt trial and error. 
Det slutade med att jag hittade någoting som fungerade. 

Jag stötte också några problem när jag försökte komma åt mina sprite bilder och json filen.
Det hände bara när jag kompilerade med cmake. Jag började med att försöka säga åt cmake vart
de ska söka efter filer, det gick inte så bra. Det slutade med att jag i själva koden skriver ut hur 
filsökvägen ser ut just i koden. Jag märkte då att när jag kompilerar i cmake så börjar 
programmet i build katalogen istället för source. Så jag ändrade koden i programmet så jag 
kommer till rätt katalog. Istället för att försöka hitta en lösning med cmake.

Jag skulle nog ha tagit hjälp av labbassisterna speciellt när det kom till första gången jag försökte 
få igång cmake. Då det fanns ett labbpass specifikt för cmake skulle jag nog kunnat spara mycket tid och lidande.

\subsection{c++ och objekt orientering}
I mitt spel så har jag en klass där största delen av all kod utförs, det vill man helst 
undvika när det kommer till objekt orienterad design så man vill att varje klass ska ha
enbart ett specifikt ansvar. En del in till programmet försökte jag fixa lite högre cohesion
med att skapa en \textbf{ObjectManager} klass. Det var inga problem tills jag stötte på ett
\textbf{SEGMENTATION} fel som jag inte alls förstod mig på. Då den inträffade på något som jag inte 
äns jag hade allokerat något minne till, det var bara en \textbf{Json::Data} variabel. 
Som funkade som den skulle tills jag lade den variablen i en if sats. 
Felet uppträffades bara när jag försökte fixa mer klasser. Det slutade med att jag gav upp 
och återvände till min low cohesion design. 

Jag hade verkligen stora svårigheter när det kom till c++. 
Jag skulle igen kunnat tagit mer hjälp från labbassisterna.
Jag tror jag också skulle ha använt IDE som clion med debug verktyg det skulle nog 
ha underlättat en del.
Då jag bara använde mig av en textredigerare för hela projektet.
Just nu hoppas jag på att inte behöve hålla på med c++ så mycket. 
Det blir java och libGDX för mig.
\section{Samarbete}
Som sagt innan så gjordes nästan hela projektet själv. Det var bara 
gruppkontraktet och första utkastet av kravspecifikationen och desingspecifikationen som 
vi samarbetade med. 

Det lilla vi samarbetade med på tillsammans var inte helt fri från "konflikter".
Då min projektpartner var väldigt upptagen med något personligt så ledde det till att 
mycket gjordes i sista minuten. Han hade också väldigt svårt att svara på mina meddelanden,
då det kunde ta upp till en vecka tills han svarade. Har varit mer än 2 veckor nu sen hans sista 
meddelande. 

Jag vet inte hur mycket jag som en individ skulle kunna påverka. 
Jag kanske skulle specifiera några åtgärder i grupp kontraket att om man inte 
är produktiv (inte svarar, bidrar) så kontaktas kursledningen eller något liknande.

Så tyvärr fick jag inte testa på hur jira, jag skulle verkligen vilja prova på hur det är 
att arbete med ''issues'' och hitta en bra metod på hur man kan tilldela kod.
Så hoppas verkligen det blir mer lyckat på mitt nästa grupparbete.
\section{Sammanfattning}
Jag skulle ändå säga att jag har lärt mig en hel del under projektet.
Då som sagt c++ och cmake är helt nytt för mig. 
Jag tyckte också det var väldigt kul att göra ett spel jag kände mig verkligen kreativ på 
en del lösningar och fick en hel del olika ideër. Så känner att jag verkligen kommer fortsätta göra spel,
kanske inte med c++ dock.
Tror inte det blir en lång karriär i c++ för mig då jag fastande i vad jag trodde skulle vara enkla lösningar 
en hel del gånger. Så det blir att satsa på ett språk med garbage collection för mig i framtiden.



\end{document}
